
\documentclass[12pt,spanish]{article}
\usepackage{babel}
\usepackage[T1]{fontenc}
\usepackage{textcomp}
\usepackage{mathtools}
\usepackage[utf8]{inputenc} % Puede depender del sistema o editor
\usepackage{multirow} % para las tablas
\usepackage{textcomp}
\usepackage{siunitx}
\usepackage{listings}
\usepackage{booktabs}
\usepackage{caption}

 
 
\title{Modelado estadístico de datos: Práctica 1}
\author{David Alarcón Rubio}
\date{Diciembre 2019}



\begin{document}
	
	\maketitle
	
	% This line here is a comment. It will not be printed in the document.
	
	\part*{Ejercicio 1.}	
	
	\textbf{1. (CALC) (2 puntos) Se ha realizado un estudio para ver si influye la metodología docente a la hora de aprobar. Para ello 50 estudiantes han recibido la metodología 1 y 50 la metodología 2. De cada estudiante se ha registrado si al final aprobaban (1) o no (2). Los datos experimentales se dan en la tabla siguiente, donde el numero de individuos con perfil aprobar = 1 y metodología = 1 es 35, con perfil aprobar = 1 y metodología = 2 es 15, con perfil aprobar = 2 y metodología = 1 es 40 y con perfil aprobar = 2 y
		metodología = 2 es 10: ¿Hay diferencias estadadisticamente significativas entre las dos metodología?}\\
	
	Calcularemos el test z de diferencia de dos proprociones que esta enmarcado en el esquema D <- D que indica que se esta intentado explicar la variable de respuesta Y (aprobados)  dicotómica a través de la variable explicativa X (metodología) también dicotómica.\\
	
	Comenzamos realizando la tabla resumen de Y=aprobar por M=metodología.\\
	
	\section*{Tabla de resultados}
	\begin{table}[h]
		\begin{tabular}{l|l|l|l|}
			\cline{2-4}
			& M=1 & M=2 & r \\ \hline
			\multicolumn{1}{|l|}{Y=1} & 35 & 15 & 50 \\ \hline
			\multicolumn{1}{|l|}{Y=2} & 40 & 10 & 50 \\ \hline
			\multicolumn{1}{|l|}{n} & 75 & 25 & 100 \\ \hline
		\end{tabular}
	\end{table}
	
	
	\section*{Ecuación}			
	\begin{equation}
	\begin{aligned}
	IC = (\frac{35}{75}-\frac{15}{25}) \pm 1.96	 \sqrt{\frac{35}{75} \cdot (1-\frac{35}{75}) \cdot \frac{1}{50}+\frac{15}{25} \cdot (1-\frac{35}{75}) \cdot \frac{1}{50}}\\
	= (0.4666 - 0.6) \pm 1.96	 \cdot 0.0988 \\= (-0.3288, 0.0588)
	\end{aligned}
	\end{equation}
	
	
	\section*{Ecuación}			
	\begin{equation}
	\begin{aligned}
	Z = \frac{\frac{35}{75}-\frac{15}{25}}{\sqrt{\frac{35+14}{75+25} \cdot (1-\frac{35+14}{75+25}) \cdot (\frac{1}{50}+\frac{1}{50}) }}\\
	= \frac{ -0.1334}{\sqrt{0.0099}} = \frac{ -0.1334}{0.0999} = -1.33
	\end{aligned}
	\end{equation}
	\\
	Podemos comprobar que |Z| = 1.33 < 1.96 luego comprobamos que no es estadísticamente significativa la diferencia de aprobados entre las dos metodologías.
	
	
	\part*{Ejercicio 2.}	
	2. (CALC) (1 punto) En el modelo de regresión lineal, se define la matriz H (matriz "hat") como aquella matriz que pone el sombrero a la y, es decir que $\hat{y}$ = Hy, entonces se verifica que H es simétrica e idempotente.\\
	
	\begin{itemize}
		\item a) Verdadero.
		\item b) Falso.
	\end{itemize}
	
	Si $\hat{y}$ = Hy, y dado que:
	
	
	\section*{Equation}			
	\begin{equation}
	\begin{aligned}
	H =  X (X^t X)^{-1}  X^t
	\end{aligned}
	\end{equation}
	
	Una matriz es simétrica si es una matriz cuadrada, la cual tiene la característica de ser igual a su traspuesta.\\ 
	
	Dado que se comprueba que H es una matriz cuadrada nXn:
	\section*{Equation}			
	\begin{equation}
	\begin{aligned}
	H =  X_{nm} (X_{mn}^t X_{nm})^{-1}  X_{mn}^t = H_{nn}\\
	Sea,  W_{mm} =  (X_{mn}^t X_{nm})^{-1}\\
	Entonces,  H =  X_{nm} W_{mm}   X_{mn}^t = H_{nn}
	\end{aligned}
	\end{equation}
\\
		 Aplicado las siguientes propiedades matriciales:
	 	\begin{itemize}
	 	\item $(B\cdot C)^t = C^t\cdot  B^t$
	 	\item $(A\cdot B\cdot C)^t=C^t\cdot  B^t\cdot  A^t$
	 	\item $(B^t)^t= B$ 
	    \item $(A^{-1})^t=(A^t)^{-1} $ 
	    \item $ A^t\cdot A=(A^tA)^t$
	 \end{itemize}

    Obtenemos que:\\
    
	\section*{Equation}			
	\begin{equation}
	\begin{aligned}
	H^t =  (X (X^t X)^{-1}  X^t)^t = (X^t)^t ((X^t X)^{-1})^t  X^t = X ((X^t X)^t)^{-1}  X^t = X (X^t X)^{-1}  X^t = H
 	\end{aligned}
	\end{equation}

	Por lo que comprobamos que H es una matriz Simétrica.\\
	

	Una matriz idempotente es una matriz que es igual a su cuadrado, es decir: A es idempotente si A × A =  $A^2$\\
	
		 Aplicado la siguiente propiedad matricial:
	\begin{itemize}
	 	\item $A\cdot  A^{-1} = I
	 \end{itemize}

    Obtenemos que:\\
    
	\section*{Equation}			
	\begin{equation}
	\begin{aligned}
	H^2 =  (X (X^t X)^{-1}  X^t)^2 =(X (X^t X)^{-1}  X^t)(X (X^t X)^{-1}  X^t)= \\
	X (X^t X)^{-1}  (X^t X) (X^t X)^{-1}  X^t = X (X^t X)^{-1} I X^t = X (X^t X)^{-1} X^t = H
 	\end{aligned}
	\end{equation}	
	
	Concluimos que la matriz H es una matriz Simétrica e Idempotente, por lo que la respuesta a la pregunta es Verdadero.
	
\part*{Ejercicio 3.}	
		\textbf{3. (CALC) (1 punto) En el modelo de regresión lineal, se define la matriz H (matriz "hat") como aquella matriz que pone el sombrero a la y, es decir que $\hat{y}$ = Hy, entonces se verica que los elementos $h_{ii}$ de la diagonal de H vienen dados por $h_{ii} = x_i^t (X^t X)^{-1} x_i$; siendo $x_i^t = (1 x_{i1} . . . x_{ip}$)}
	
	
	\begin{itemize}
		\item a) Verdadero.
		\item b) Falso.
	\end{itemize}
	
	
 Siendo: 	$x_i^t = (1 x_{i1} . . . x_{ip}$)
 Entonces: 

 	\begin{equation}
		\begin{aligned}
		X_{np+1}^t =
		%
\begin{pmatrix}
1 & x_{11} & . . .& x_{1p} \\
1 & x_{21} & . . .& x_{1p}\\
... & ...& ...   & ...\\
... & ...&  ...  &  ...\\
1 & x_{n1} & . . .& x_{np}\\
\end{pmatrix}
 	\end{aligned}
	\end{equation}
	
	
	
	
					
	
	Si se denota por $(q_{ij}) = (X^tX)^{-1}$, que tiene dimensión (p + 1) x (p + 1) y se realiza el producto matricial, se tiene que:\\
	
	\begin{equation}
		\begin{aligned}
	H = %
\begin{pmatrix}
1 & x_{11} & . . .& x_{1p} \\
1 & x_{21} & . . .& x_{2p}\\
... & ...& ...   & ...\\
... & ...&  ...  &  ...\\
1 & x_{n1} & . . .& x_{np}\\
\end{pmatrix}
% 
\begin{pmatrix}
 q_{11} &  q_{12}&. . .& q_{1p} \\
 q_{21} &  q_{22}&. . .& q_{2p} \\
... & ...& ...   & ...\\
... & ...&  ...  &  ...\\
 q_{p+11} &  q_{p+12}&. . .& q_{p+1p+1} \\
\end{pmatrix}
%
\begin{pmatrix}
1 & 1  & . . .& 1 \\
x_{11} & x_{21} & . . .& x_{n1}\\
... & ...& ...   & ...\\
... & ...&  ...  &  ...\\
 x_{1p} & x_{2p} & . . .& x_{np}\\
\end{pmatrix}
\\
	 = %
\begin{pmatrix}
 x^t_1 q_{11} &  x^t_1 q_{12}&. . .& x^t_1 q_{1p} \\
x^t_2 q_{21} & x^t_2 q_{22}&. . .& x^t_2 q_{2p} \\
... & ...& ...   & ...\\
... & ...&  ...  &  ...\\
x^t_n q_{p+11} & x^t_n q_{p+12}&. . .& x^t_n q_{p+1p+1} \\
\end{pmatrix}
%
\begin{pmatrix}
1 & 1  & . . .& 1 \\
x_{11} & x_{21} & . . .& x_{n1}\\
... & ...& ...   & ...\\
... & ...&  ...  &  ...\\
 x_{1p} & x_{2p} & . . .& x_{np}\\
\end{pmatrix}
 	\end{aligned}
	\end{equation}
	
Por lo que: \\

\begin{equation}
		\begin{aligned}
	h_{ii} = %
\begin{pmatrix}
 x^t_i q_{11} &  x^t_i q_{12}&. . .& x^t_i q_{p+1} \\
\end{pmatrix}
% 
 x_i =  x^t_i %
\begin{pmatrix}
 q_{11} &  q_{12}&. . .& q_{p+1} \\
\end{pmatrix}
%  
 x_i =  x_i^t (X^t X)^{-1} x_i

 	\end{aligned}
	\end{equation}
	

\part*{Ejercicio 4.}	
		\textbf{4.(CALC) (2 puntos) El siguiente código en R}\\
\begin{lstlisting}[language=R]
rm(list=ls())
datos=read.table('c_d_1.txt',header=T)
attach(datos)
ind1=which(exp==1)

ind2=which(exp==2)
n1=length(rta[ind1]); n1
n2=length(rta[ind2]); n2
tapply(rta,exp,mean)
tapply(rta,exp,sd)
t.test(rta[ind1],rta[ind2],var.equal=TRUE)
\end{lstlisting}	
\\
proporciona el siguiente resultado\\


\begin{lstlisting}[language=R]
> n1=length(rta[ind1]); n1
[1] 7
> n2=length(rta[ind2]); n2
[1] 10
> tapply(rta,exp,mean)
1 2
25.85714 26.20000
> tapply(rta,exp,sd)
1 2
9.856108 8.866917
Two Sample t-test
data: rta[ind1] and rta[ind2]
t = -0.075009, df = 15, p-value = 0.9412
alternative hypothesis: true difference in means is not equal to 0
95 percent confidence interval:
-10.085497 9.399783
sample estimates:
mean of x mean of y
25.85714 26.20000
\end{lstlisting}
\\
A continuación se escribe el siguiente código:\\

\begin{lstlisting}[language=R]
exp2=1*(exp==2)
summary(lm(data = datos,formula = rta ~ exp2))
\end{lstlisting}

que proporciona el siguiente resultado.

\begin{table}[]
\begin{tabular}{@{}lllll@{}}
\toprule
            & Estimate & Std. Error & t value & Pr(\textgreater jtj) \\ \midrule
(Intercept) & xxx      & xxx        & xxx     & xxx                  \\
exp2        & xxx      & xxx        & xxx     & xxx                  \\ \bottomrule
\end{tabular}
\caption{Tabla 2: Coeficientes de RL con p = 1 sin informacion rellenada}
\end{table}
\\
Residual standard error: xxx on xxx degrees of freedom \\
Multiple R-squared: xxx, Adjusted R-squared: xxx \\
F-statistic: xxx on xxx and xxx, p-value: xxx \\
Se pide rellenar el mayor numero posible de valores marcados con xxx. \\



\part*{Ejercicio 5.}	
		\textbf{4.(CALC) (2 puntos) El siguiente código en R}\\
\begin{lstlisting}[language=R]
rm(list=ls())
datos=read.table('c_n_1.txt',header=T)
attach(datos)
ind1=which(exp==1);

ind2=which(exp==2);
ind3=which(exp==3);
n1=length(rta[ind1]); n1
n2=length(rta[ind2]); n2
n3=length(rta[ind3]); n3
tapply(rta,exp,mean); tapply(rta,exp,sd)
summary(aov(rta~factor(exp)))
\end{lstlisting}	
\\
proporciona el siguiente resultado\\


\begin{lstlisting}[language=R]
> n1=length(rta[ind1]); n1
[1] 7
> n2=length(rta[ind2]); n2
[1] 10
> n3=length(rta[ind3]); n3
[1] 5
> tapply(rta,exp,mean); tapply(rta,exp,sd)
1 2 3
25.85714 26.20000 22.60000
1 2 3
9.856108 8.866917 8.876936
Df Sum Sq Mean Sq F value Pr(>F)
factor(exp) 2 46.7 23.35 0.276 0.762
Residuals 19 1605.7 84.51
\end{lstlisting}
\\
A continuación se escribe el siguiente código:\\

\begin{lstlisting}[language=R]
exp2=1*(exp==2)
exp3=1*(exp==3)
summary(lm(data=datos, formula=rta ~ exp2+exp3))
\end{lstlisting}

que proporciona el siguiente resultado donde se pide rellenar el mayor numero posible de valores marcados con xxx.

\begin{table}[]
\begin{tabular}{@{}lllll@{}}
\toprule
            & Estimate & Std. Error & t value & Pr(\textgreater jtj) \\ \midrule
(Intercept) & xxx      & xxx        & xxx     & xxx                  \\
exp2        & xxx      & xxx        & xxx     & xxx                  \\
exp3        & xxx      & xxx        & xxx     & xxx                  \\ \bottomrule
\end{tabular}
\caption{Tabla 2: Coeficientes de RL con p = 2 sin informacion rellenada}
\end{table}


Residual standard error: xxx on xxx degrees of freedom 
Multiple R-squared: xxx, Adjusted R-squared: xxx 
F-statistic: xxx on xxx and xxx, p-value: xxx 






\part*{Ejercicio 6.}	
		\textbf{6. (2 puntos) Se ha realizado un estudio para ver si el peso en kg (rta) de unos deportistas depende de su cintura en cm (exp1), del numero de km de entrenamiento (exp2) y del tipo de entrenamiento (exp3=1: Body building, exp3=2: Fitness). Han participado en el estudio 26 individuos. Los datos experimentales estan en el chero c ccd.txt alojado en el curso virtual y se muestran en la tabla 6.
Se pide:
Interpretar los resultados del modelo de regresion lineal con todas las variables.
Repetir el analisis quitando las variables no signicativas. >Que sucede?
Crear una variable interaccion entre exp1 y exp3 e incorporarla al modelo anterior. >Que ocurre?
Elegir de los tres modelos anteriores el mejor. >Se cumplen las condiciones de aplicabilidad de la
regresion lineal?
Elaborar otro enunciado para estos datos.
En el documento que se entregue habra que incluir el codigo utilizado.}





\begin{table}[]
\begin{tabular}{@{}llll@{}}
\toprule
rta  & exp1 & exp2 & exp3 \\ \midrule
69.3 & 83   & 8    & 1    \\
69.6 & 84   & 7    & 1    \\
71.5 & 86.5 & 4    & 1    \\
71.5 & 84.5 & 32   & 1    \\
70.6 & 86.4 & 15   & 1    \\
69.2 & 82.5 & 6    & 1    \\
65   & 82   & 10   & 2    \\
65.4 & 81.8 & 17   & 2    \\
63.7 & 80   & 6    & 2    \\
69   & 82.5 & 18   & 1    \\
65.8 & 84   & 0    & 2    \\
68.7 & 87.2 & 3    & 2    \\
64.8 & 84   & 10   & 2    \\
70   & 86   & 11   & 1    \\
65.9 & 84.2 & 18   & 2    \\
63.9 & 84   & 4    & 2    \\
62.1 & 79   & 12   & 2    \\
73.1 & 97.2 & 18   & 2    \\
75.4 & 91   & 0    & 1    \\
72.6 & 89.5 & 9    & 1    \\
69.6 & 89.5 & 11   & 2    \\
72.3 & 87.5 & 7    & 1    \\
67.3 & 87.5 & 15   & 2    \\
68   & 87.5 & 5    & 2    \\
68.1 & 86.5 & 14   & 2    \\
71.3 & 87   & 9    & 1    \\ \bottomrule
\end{tabular}
\end{table}


\end{document}